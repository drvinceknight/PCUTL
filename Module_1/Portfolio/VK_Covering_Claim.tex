\documentclass[12pt]{article}

\setlength{\parindent}{0pt}
\setlength{\parskip}{.5cm}
\setlength{\textheight}{220mm}
\setlength{\textwidth}{166mm}
\setlength{\topmargin}{0mm}
\setlength{\oddsidemargin}{0mm}


\title{PCUTL - Module 1:\\ Covering Claim}
\author{Vincent Knight}
\date{\today}

\begin{document}

\maketitle

The ILOs for this journal are:
\begin{itemize}
\item Describe the local and national contexts with respect to UK HE policy, bother generically and in their subject, and consider their role(s) within it.
\item Plan and run sessions that support student learning by giving active roles to students, fostering critical and independent thinking according to the standards of their subjects.
\item Explore the relationship between research, scholarship, related professional activities and teaching and learning as relevant to their own teaching practice.
\item Use PRLT to explore the impact of their teaching and/or support for learning on students' learning, and plan modifications accordingly.
\item Identify further professional development needs in relation to teaching and/or supporting student learning.
\end{itemize}

This portfolio contains a variety of materials. Firstly an annotated lesson plan is presented. This lesson plan was for a particular lesson that I enjoy giving and which lends itself well to giving active roles to students. My mentor has given a detailed review of this plan following which I have responded with ideas for how I am going to improve my teaching. Due to the timing of this lesson, I completed it before my reflective journals which I feel allowed me to reflect on the type of teacher I would like to be with relation to the discussion that occurred following the peer review. Before the appendices that include various teaching materials (as well as a page with detailed links to online materials) I include a short document that summarises further reflection that occurred as a result of discussions with my mentor. The role for that document is to give an outlook of how this portfolio will/can be carried on to Module 2.

In my first reflective journal I feel that I have addressed ILO 1 as well as ILO 3. The third ILO was further addressed in detail in the other reflective journals where I consider the various ways in which I learn, what I consider a teaching to be and finally what this implies for my teaching.

ILOs 2 and 4 are naturally addressed by the PRLT. As should be evident in my response to the peer review I have specifically explained how I plan to make certain modifications to my teaching.

Finally ILO 5 is addressed by the various discussions had with my mentor and PCUTL staff. I will concentrate on my capacity to mainstream the certain teaching methodologies as well as exploring other teaching models.

Through the various reflections that form that major part of this module (be it through the reflective journals or through the mentor peer review) it has become clear to me that I think of good practice in teaching as being able to ``create learning opportunities". This is a theme that I plan to emphasise and develop further as I progress through the various PCUTL modules. I hope to improve my own teaching of mathematics so as to ensure that I always provide students with the best learning opportunities that I am able to. Throughout PCUTL I hope to further understand a variety of learning and teaching modules as well as ensure that I am aware of a variety of technological solutions that ensure I give active roles to students. One way I plan to do this is to evaluate how students have responded to the various opportunities I presented them during my teaching this term (Autumn 2012).



\end{document}
