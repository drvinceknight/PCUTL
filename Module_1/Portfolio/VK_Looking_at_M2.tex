\documentclass[12pt]{article}

\setlength{\parindent}{0pt}
\setlength{\parskip}{.5cm}
\setlength{\textheight}{220mm}
\setlength{\textwidth}{166mm}
\setlength{\topmargin}{0mm}
\setlength{\oddsidemargin}{0mm}


\title{Some concluding thoughts}
\date{}


\begin{document}

\maketitle

I hope that the main theme of PCUTL for me will be to ensure that I ``create learning opportunities''. This term (Autumn 2012) I taught the first four weeks of a module that taught a variety of analytical methods to MSc students. Throughout this period I ``gave a variety of opportunities'' to students through the use of videos, computer programs and teaching notes that were available to students before and after lectures. My students were kind enough to fill in some feedback with regards to how they used these resources. I look forward to analysing this further in Module 2. Another teaching model I hope to explore is the use of ``Inquiry Based Learning'' or ``Problem Based Learning''. One way I plan on doing this is through the use of simple problems distributed to students prior to a lesson. I'll ask students to present their solutions to the problem to the rest of the class at the beginning of each lecture. There are various goals to this:

\begin{itemize}
\item Improve their presentation skills.
\item Make active learning of the students.
\item Make students more aware of the learning opportunities available to them through a hybrid of the classic ``flipped classroom''/``Inquiry Based Learning'' models.
\end{itemize}


\end{document}
