\documentclass[a4paper,12pt]{article}

\title{PCUTL - Response to Module 1 Feedback}
\author{Vincent Knight}
\date{\today}

\begin{document}

\maketitle

In this document I will respond to each element of feedback from my module 1 portfolio.

\section{Learning Outcome One}

\begin{quote}
    ``Your portfolio is soundly located within a range of UK and MATHS specific agenda. You adopted a spiral structure to your journals that enabled you to think, apply, rethink ect. But did you ever actually arrive at a position? Where are you at now in your thinking about all these ideas and agenda may mean for your creation of opportunities for students to learn?
\end{quote}

I think I agree with this comment: I did not necessarily arrive at a robust position.\\

I found this learning outcome and the whole process of reflective writing difficult. I enjoy structure in my work and felt that I had to `let go' to make progress in my portfolio. Perhaps as a result I lost direction.\\

Having said that I certainly feel that the process was very beneficial to myself as an educator. In particular it helped me focus a bit more on my teaching methodologies in Module 2. I feel that I have in Module 2 been able to arrive at a robust position.

\section{Learning Outcome Two}

\begin{quote}
    ``You created a lesson plan specifically to engage students in their learning and have done well to also explore how Kolb's ideas might work in your context. By the end of the journals you came back to the plan and suggested some very appropriate adjustments - well done!''
\end{quote}

I appreciate the kind words of feedback here. I put a lot of effort in to this lesson plan and thanks to various discussions with my mentor I was able to fine tune it further.\\

Having said that I feel that in this module I have been able to conceptualise and justify the approaches I use as well as find further ways of placing the learning opportunities I try to create in a constructivist framework.

\section{Learning Outcome Three}

\begin{quote}
    ``The portfolio demontrates your deepening engagement with some generic and MATHS specific literature. In Module 2 have a look at other pedagogies and perhaps also the writing of inclusive and level appropriate ILOs - something like Biggs' SOLO taxonomy may take you helpfully out of your comfort zone.''
\end{quote}

I certainly feel that I have left my comfort zone in this module! I have taken this advice to heart and invested a lot of time in to learning about a variety of pedagogies. Exploring phylosophical and psychological ideas is not something I expected to enjoy but having said that I've thouroughly enjoyed it! It's been very rewarding to learn about theories of learning and reflect with these on various teaching styles that I feel are appropriate to me. (I also found a mapping of Biggs' taxonomy to mathematics that could be a useful resource for future Mathematics PCUTLers).

\section{Learning Outcome Four}

\begin{quote}
    ``You and your mentor have worked well together to discuss your sessions. In Module 2 use your PRLT to seek an outsider's perspective and draw on the literature to support/contest your observations.''
\end{quote}

I agree that the PRLT with my mentor was valuable. I have taken advantage of my PRLT in this module and enjoyed the perspective offered by my peer. I've also investigated as much of the literature as I was able to in the time allowed.

\section{Learning Outcome Five}

\begin{quote}
    ``The structure of your portfolio has enabled you to revisit some ideas a number of times and demonstrates your deepening `noticing' and plans for development - excellent!''
\end{quote}

I agree that the reflective part of the previous portfolio allowed me to `notice' a variety of things about my teaching. I've enjoyed being able to justify these things in this portfolio.

\section{Programme Values}

\begin{quote}
    ``Your practice is clearly underpinned by a commitment to facilitating student learning. As you go through PCUTL perhaps extend your ideas about learning communities and create a group to discuss teaching among your MATHS colleagues perhaps?''
\end{quote}

I feel that I've invested alot in to building and participating in learning communities amongst my peers (details of this are given in my PMITT document). I have not been able to create a group in MATHS however will give this some more thought.

\section{Engagement with the UKPSF}

\begin{quote}
    ``The map provide good evidence of your explicit engagement with the UKHE teaching agenda.''
\end{quote}

I have hopefully been able to do the same for this module.

\section{General coomment/thoughts for Module 2}

\begin{quote}
    ``Well done Vince! This has been an honest and carefully conceived submission. In Module 2 take the opportunity to move out of your safe (Kolby) spaces and explore ideas that support peer learning.''
\end{quote}

I appreciate the feedback and feel that I've managed to step out of my comfort zone (as described previously). I'm really beginning to enjoy the process (I must admit that I found the reflective journals in the previous module difficult) and look forward to Module 3!

\end{document}
