\documentclass[12pt,a4paper]{article}

\usepackage{todonotes}

%\setlength{\parindent}{0pt}
%\setlength{\parskip}{.5cm}
%\setlength{\textheight}{220mm}
%\setlength{\textwidth}{166mm}
%\setlength{\topmargin}{0mm}
%\setlength{\oddsidemargin}{0mm}

\usepackage{enumitem}


\begin{document}

\begin{center}
\Huge{VK - \textbf{Annotated} Game Theory Annotated Lesson Plan}
\end{center}

\section{Description}

This is a lesson plan for the afternoon ``teaching period'' which aims to cover the Game Theory part of the syllabus of the MAT001: OR Methods MSc module.
\todo[inline]{Annoted to take in to account inclusivity from workshop: In blue: the essential academic competences; In red: potential barriers; In blue: example of good practice; In orange: potential improvements.}
The technical content of the module aims to cover the following topics:

\begin{itemize}
\item Normal Form Games
\item Pure Nash Equilibrium
\item Mixed Nash Equilibrium
\end{itemize}

The intended learning outcomes for this module.

On completion of the module a student should be able to:

\begin{enumerate}[label=\Alph*]
\item Describe a general appreciation of the ideas of game theory.
    \todo[inline,color=green!40]{Basic academic skills required: reading}
\item Interpret the normal form of a game.
    \todo[inline,color=green!40]{Basic academic skills required: comprehension}
\item Use ideas such as common knowledge of rationality and dominance to identify dominated strategies.
    \todo[inline,color=green!40]{Basic academic skills required: comprehension and basic computation}
\item Identify best responses to certain strategies in games.
    \todo[inline,color=green!40]{Basic academic skills required: comprehension and basic computation}
\item Use the two above skills to identify pure Nash equilibria in games.
    \todo[inline,color=green!40]{Basic academic skills required: comprehension and basic computation}
\item Describe the basic ideas of mixed strategies.
    \todo[inline,color=green!40]{Basic academic skills required: comprehension and basic computation}
\item Compute mixed strategy equilibria using the equality of payoffs theorem.
    \todo[inline,color=green!40]{Basic academic skills required: comprehension and basic computation}
\item Describe an appreciation of the relationship of state of the art research and the topics they have learnt.
    \todo[inline,color=green!40]{Basic academic skills required: comprehension}
\end{enumerate}

\section{Lesson plan}
\begin{center}
\begin{tabular}{|p{2cm}|p{.75cm}|p{4.5cm}|p{4.5cm}|p{2cm}|}
\hline
Time&ILO&Teacher Activity& Learner Activity& Resources\\\hline
Before the class&A,H&Invite student to look at videos:\begin{itemize}\item EU-EMS Interface \item Introduction to mixed strategies\end{itemize}&View videos\todo[inline,color=red!40]{Potential issues with comprehension and/or access to internet}&Videos\\\hline

0-20mins&A,C, D,E&Explain 2/3rds of average game&Play 2/3rds of average game\todo[inline,color=red!40]{Language or other reading disability}&Slides + Handouts\\\hline
20-40mins&B,C, D,E&Lecture on Normal Form Games and Pure Strategies&Listen\todo[inline,color=red!40]{Language or other reading disability}&Slides + Videos\\\hline
40-50mins&NA&Break&Break&NA\\\hline
50-80mins&A,D, E&Run PD tournament&Active participation in PD tournament:\begin{itemize}\item Group discussion of stratgies\item Inter group discussion of strategies\item Duel\end{itemize}\todo[inline,color=red!40]{Language or other reading disability}&Playing Cards\\\hline
80-100mins&F,G&Lecture on Mixed Strategies&Listen\todo[inline,color=red!40]{Language or other reading disability}&Slides + Sage Interact\\\hline
100-130mins&F,G&Help if needed& In groups of 2 work on Sage lab sheet\todo[inline,color=red!40]{A part from issue with language barriers within a group I don't foresee any potential barriers.}&Sage lab sheet and interact\\\hline
\end{tabular}
\end{center}
\todo[inline,color=red!40]{Main potential barriers: comprehension due to language and reading of lecture material.}

\section{Assessment}
The above ILO would be assessed as part of the wider assessment for MAT001: part of the exam. This type of assessment is well suited to the subject area as it will allow for evaluation of each ILO.
\todo[inline,color=red!40]{The standard barriers are to be expected when it comes to a written exam.}


\section{Comments}
This lesson plan is designed in such a way as to ensure that most learning styles will be catered for. Indeed students are initially introduced to Game Theory through a game itself. This should allow students to be put in the place of a decision maker and thus allow for the more complex concepts to be easier to understand. Furthermore, it is hoped that this will relax students and make way for better students participation throughout the course.

To complement the two participation activities there are two breaks scheduled which are put in place to ensure that students won't be tired and their concentration will not falter. Apart from these activities information will also be delivered through two short lectures and there is also a group exercise scheduled which will hopefully invite peer learning. This exercise is intentionally different to all other learning exercises presented in this period. It makes use of programming and relates to a video that students will hopefully watch before the lecture.

Finally there is a tutorial session scheduled that will allow for students to carry out classical pen and paper exercises which should help them prepare for the exam. This tutorial will be taken by 2 postgraduate students to hopefully offer a different perspective however I will still be present in case I am needed.

\todo[inline,color=blue!20]{Examples of good practice: The main potential barriers for this lesson plan are with respect to communication. The prior viewing of videos and sharing of course material should help with this issue. They allow the students to see the content prior the class and of course ``listen, read" at their pace so that if a student did not understand a particular point they have the time to ``re-listen, re-read". I give as much content to my students as possible prior to the lecture, I do this to ensure that they have as much time as possible to familiarise themselves with the content.

I also feel that I endevour to speak clearly and the rooms are equipped with audio feedback loops so students can use them if they choose.
}

\todo[inline]{Potential improvements: All of my notes are delivered to the students in pdf format. This is by design. I want my students to have access to the content anywhere without the need for proprietary software (all my teaching is done with open source software). The pdf's allow students to view the notes where they want them and also to zoom as necessary. The source code for all my notes are also available online so that if need be the students could change the notes to make them easier to read. Having said that one potential improvement would be to give students notes in multiple formats so that they could read them as they wished: html, pdf, md, docx etc... This is easily done using pandoc.}

\end{document}
