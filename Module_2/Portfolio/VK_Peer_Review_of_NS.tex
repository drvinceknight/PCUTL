\documentclass[a4paper,12pt]{article}

\title{PCUTL - Module 2:\\ Peer Review of Nikos}
\author{Vincent Knight}
\date{\today}

\begin{document}

\maketitle

\section{General comments}

I think that this is a well thought out lesson plan with an excellent set of well written notes. I personally feel that it's great if students see notes befoer a lecture and as such like that Nikos plans to give the notes to the students.

Before going on to explore particular aspects of the lesson plan there is one aspect that I thought was particularaly worth highlighting. Nikos will be using a qusi immediate feedback system that he calls the `muddiest point'. This is a great idea and something I might think of implementing myself.

\section{Inclusivity}

Nikos has taken care to consider various issues of inclusivity with regards to access to the room and other issues of communication. One minor suggestion that I would have is with regards to the formats of the notes. It is obvious that because of the diagrams and mathematical notation the notes need to be prepared in \LaTeX\ and distributing them in pdf format is natural. Perhaps to take in to account the fact that some students might have difficulties with reading the notes Nikos could also make the \LaTeX\ code available so that if need be students could modify the notes to make them easier to read?

\section{Contact and non contact time}

Some great effort has been put in place to consider contact and non contact time. My only concern is that because Nikos has (very rightfully!) designed the plan so that student who do not look through any of the material will not be excluded from learning, will any students in fact look through the material?\\

Perhaps some sort of incentive could be put in place to ensure students take advantage of the efforts made by Nikos to create non contact time learning opportunities. I feel that this would ensure that students are placed in a constructivist learning model. I might be incorrect and Nikos would rather teach allowing for a constructivist model if the student chose to use it whilst ensuring a cognitivist model otherwise.\\

Another aspect that would be worth considering is the use of an online discussion board? Perhaps one of these could be setup on Learning Central (I'd recommend the campus pack one).

\section{Technology enhanced education}

Great use of video in and out of contact time. The notes are of high quality and the \LaTeX\ code is not basic as the Tikz package is used. My only other technologic suggestion would be perhaps setting up a discussion board?


\section{Suggestions and conclusions}

Good work getting the students to go through the Kolb cycle and learning in various ways.\\

The course seems to be set in a cognitive setting. Despite asking students to attempt the one example. Perhaps an incentive asking students to do that would be better?\\

I would also suggest that where possible Nikos should insist that students work in groups to add a social learning dimension to his lesson.\\

On a whole those comments are minor as I feel that this is a well thought out lesson plan.\\

\end{document}
