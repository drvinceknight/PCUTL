\documentclass[a4paper,12pt]{article}

\usepackage{hyperref}
\usepackage{graphicx}

\title{PCUTL - Module 2:\\ Covering Claim}
\author{Vincent Knight}
\date{\today}

\begin{document}

\maketitle

In this short covering claim I will describe my portfolio for this module in general as well as how I meet the various ILOs.\\

This portfolio contains the following documents (in order):

\begin{itemize}
    \item My response to feedback for Module 1;
    \item ``Pedagogic Models, Inclusive Teaching and Technology'' (PMITT);
    \item My annotated Module 1 lesson plan;
    \item My lesson plans for Module 2:
        \begin{itemize}
            \item A detailed plan for an MSc module;
            \item 4 detailed lesson plans for that module.
        \end{itemize}
    \item A page with a url for all relevant teaching resources;
    \item Nikos Savva's peer review of the above lesson plans with my response;
    \item My peer review of Nikos Savva.
\end{itemize}

The main body of work for this module is contained in the second document (PMITT). In fact that document was originally going to be the covering claim for this portfolio, unfortunately the word count for that document began to grow quite substantially as I continued through my journey. As such, I will now describe the various parts of my portfolio highlighting how I have met the prescribed ILOs and where the various parts of interest can be located.\\

It is my understanding that one of the aims of the ILOs for this module aims to lead ``us'' (PCUTL students) down the road of innovation. As described throughout my Module 1 portfolio (which can be found here: \url{http://www.vincent-knight.com/home/teaching/pcutl}) I was already quite innovative in my classroom. In particular through the use of videos in pre and post contact time. Based on this, this module has been of very high value to myself as it has allowed me to situate my methods in an evidenced framework. Furthermore, thanks to a detailed analysis of the literature and conversations with my peers, I've been able to identify and put in place further innovations.\\

The starting point for my portfolio is the analysis of feedback for the MAT001 module that I used for my Module 1 PRLT. This is located in Section 1 of PMITT. The feedback was designed in such a way as to evaluate how the students engaged with certain technological innovations that I use in my teaching such as social networks and videos. \textbf{This section of my portfolio addresses ILOs 3 and 4 as I use it to evaluate the effectiveness of some of my teaching methods. I also identify certain things that can be improved which lead me on to further exploration.}\\

In Module 1 I considered inclusivity and diversity briefly but not in a detailed way which I was able to do in this module. In particular I've addressed a particular aspect of inclusivity concerning the distribution of teaching resources which was something I had not considered prior to this module. This is described in Section 2 of PMITT. \textbf{This section of my portfolio address ILO 2 as I use it to design methods to take account of individual differences in learners. In particular I design a technique for creating inclusive mathematical notes through the use of technology.}\\

The next section (3) of PMITT is a detailed look at Pedagogic models. I start by going back to basics so to speak and describe three models (Behaviouralism, Cognitivism and Constructivism). Further to this I mapped these models on to the teaching of mathematics was able to place myself in a social constructive framework. Formalising my teaching approach was valuable as I was able to evidence my previous methods as well as see what was missing. In this section I also describe cultural and social learning and decide on ways to improve my teaching by incorporating more group work. \textbf{This section very addresses ILOs 1, 2 and 3 as I clearly evidence a broad understanding of pedagogic models as well as identify what their definition implies for my teaching. I also discuss the use of technology when talking about flipped classrooms which correspond to non-contact time learning opportunities. Being able to identify myself within a well understood teaching model has helped me to improve my teaching through the use of group work and formalised scaffolding in Vygotsky's ZPD.}\\

Section 4 of PMITT is a detailed discussion of my involvement in learning communities. I discuss my engagement with the PCUTL discussion boards, my use of social networks and also my participation in a HEA workshop. \textbf{This section partly address ILO 3 as it shows how I've drawn on interaction with my peers to improve my teaching.}\\

The final section of PMITT identifies various ways in which I can continue to develop as a teacher. \textbf{This section addresses ILO 4.}\\

To summarise, this portfolio and PMITT in particular is an account of my journey through Module 2. I began this module having already some experience of using technology in an innovative way however this was not based on evidence of good practice. Through this module I have not only been able to find such evidence but also improve my methodologies in various ways such as developing methods of creating inclusive mathematical notes and incorporate more group work in my teaching. I feel that I end this module as a much more complete teacher with a better understanding of teaching and learning.\\

I look forward to finding the next things that I need to understand.

\end{document}
