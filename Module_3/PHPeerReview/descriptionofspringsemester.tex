\documentclass{article}
\usepackage{graphicx}
\usepackage{tikz}
\usetikzlibrary{intersections}
\usepackage{amssymb,amsmath}
\usepackage{hyperref}

\usepackage{fullpage}
\usepackage[parfill]{parskip}

%\setlength{\parindent}{0cm}


\title{Computing for Mathematics: description of spring semester for peer review}
\author{Vincent Knight}
\date{}

\begin{document}

\maketitle

\section{Context}

The main focus of this semester is the fourth ILO of the course:

\begin{quote}
    ``Work in groups to tackle problems and convey solutions to those problems through presentation."
\end{quote}

To enhance the student experience and ensure that entrepreneurial skills are embedded in the curriculum, this will be done in an entrepreneurial setting:

\begin{itemize}
    \item Students will be working in groups: referred to as `companies'.
    \item Companies will carry out a form of market research to ensure they choose a topic that is relevant.
\end{itemize}


\section{Delivery}

The delivery of this part of the module will not involve a lot of contact time. Students will be expected to self regulate their groups and meet often.

    \subsection{Non Contact}

        \begin{itemize}
            \item Students to form groups of 4 (this should be done by the end of week 1);
            \item Group roles to be assigned: project manager (pm) and secretary. It is the role of the pm and the secretary to ensure that the companies meet twice a week and that minutes are kept.
        \end{itemize}

    \subsection{Contact}
        \begin{itemize}
            \item In weeks 1 to 6 there will be one lecture a week. During this lecture the following themes will be covered:
                \begin{itemize}
                    \item General entrepreneurship (by the Cardiff University Union entrepreneurship team);
                    \item Library skills (by the Helen Staffer: Cardiff University Librarian);
                    \item Market strategy (by the Cardiff University Union entrepreneurship team);
                    \item Presentation skills (by the Cardiff University Union entrepreneurship team);
                    \item Meeting skills (by the Cardiff University Union entrepreneurship team).
                \end{itemize}
            \item In week 6 pm's will attend a `grand council' to give a short (less than a minute) presentation to each other. This will serve so as to encourage social learning between the groups and so that students know what else is happening in the class.
            \item In weeks 7 to 11, there will be a series of seminars instead of lectures. These will be used by invited speakers to discuss other languages but also in some cases I will ask students to potentially present to each other. In future years I will ask past students to present their project.
        \end{itemize}

\section{Assessment}

There are two forms of assessments this semester:

\begin{itemize}
    \item A short report: `project proposal' to be handed in at the end of week 3. This will be summative however the main role of this assessment is formative. It will allow me to ensure that students are on the right track and give feedback to the groups.
    \item A group presentation to be given in week 11, during this presentation students will present their work.
\end{itemize}

See the marking criteria and related documents for more information about this.

\end{document}
