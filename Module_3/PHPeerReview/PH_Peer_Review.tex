\documentclass[a4paper,12pt]{article}

%\setlength{\parindent}{0mm}
%\renewcommand{\baselinestretch}{1.15}
%\setlength{\parskip}{3mm}
\usepackage{fullpage}
\usepackage[parfill]{parskip}

\title{PCUTL - Module 3:\\ Peer Review of Vincent Knight}
\author{Prof. Paul Harper}
\date{\today}

\begin{document}

\maketitle

\section{Context}

This peer review specifically address ILO's 1, 2, 3 and 6 of the PCUTL module 3, namely to:
\begin{itemize}
\item Integrate scholarship, research and professional activities with teaching and supporting learning [ILO 1];
\item Design and plan effective modules or clusters of sessions or programmes of study that facilitate quality learning and the achievement of appropriate learning outcomes by a range of learners [ILO 2];
\item Design and implement appropriate and effective assessment and feedback schemes using a range of methods that align with the tenets and principles of the Cardiff University Assessment Strategy and Feedback Policy [ILO 3];
 \item Work with colleagues to enquire critically into an aspect of planning or assessment / feedback relevant to their context [ILO 6].
\end{itemize}

Specifically, the peer review focussed on planning for a new module that Vince is teaching to our first year undergraduate Mathematics students, MA1003 Computing for Mathematics (CfM), which is comprehensively discussed in his module 3 portfolio.  This module covers both Autumn and Spring semesters, the Autumn semester which is being taught within a social constructivist framework to encourage student learning via a flipped classroom approach.  Might I add in passing how impressed I have been by the level of effort and considered planning that has gone into this module by Vince, and the innovative methods being employed.  At the time of writing, the first couple of weeks have been hugely successful and indeed Vince is \emph{creating learning opportunities} for students as very much his desire and indeed aligned to his overall theme for PCUTL.

For this particular peer review, we discussed planning for the Spring semester which is designed to meet the 4th ILO for MA1003:

\begin{itemize}
  \item Work in groups to tackle problems and convey solutions to those problems through presentation.
\end{itemize}

The focus of the semester is on the use of programming skills in an entrepreneurial environment.  Students will be organised into groups of 4 and these groups will be called \emph{Companies}. Companies will contain a \emph{Project Manager} and a \emph{Secretary} (selected from amongst the members).

There is a limited amount of contact time in the Spring semester of CfM with one session (lecture or seminar) timetabled each week. At the beginning of the semester this will be used by the entrepreneurial team from the Cardiff University Students’ Union, and Librarians, to present key skills to the students. The seminars will be used to expose students to further aspects of programming such as other programming languages and/or some research that is being done in the school.

The work carried out this semester will be assessed using group coursework (three week report contributing 5\% towards the overall module) as well as a group presentation (contributing 25\%).


\section{Critique}

Vince and myself discussed at length various aspects of his plans for this part of CfM.  These may be  summarised under the following headings, where a critique of each aspect is provided.

\textbf{Group membership, dynamics and inclusivity}

Vince makes great efforts to ensure his teaching methods and materials are inclusive. We discussed how groups (companies) will be formed, either through self-selection or pre-defined groupings set by Vince, with an aim to create inclusive groupings. For other modules with group structures (such as the focus in previous PCUTL modules around MAT013) these have largely been constructed by our Knowledge Transfer Officer (Mrs Joanna Emery) to ensure a mix of ages, gender and cultural heritage etc. Vince needs to give more thought to this matter and if this approach is practical with 40+ groups or whether self-selection is acceptable (and perhaps look again at related educational literature for insights).

A further issue with a module of this nature (computer programming) is how satisfactory group dynamics and an equitable workload might be achieved given for example some members may have natural strengths in coding whereas others are much weaker (computer programming does tend to polarise students!). These are mostly addressed through the allocation of project manager and secretarial roles to oversee group progress, meeting regularly (twice weekly at least with minutes taken), the 3 week progress report, the week 6 `Grand Council', and that there are plenty of other activities required to take place alongside the coding itself (market research, report writing, presentation etc.). These are all excellent mechanisms that Vince has put in place, and coupled with the timetabled sessions to support student learning, one would hope for good group dynamics. Nevertheless there is potential, as always, for concerns on equitable contribution by all group members and Vince might give more thought to how to address and rectify such concerns that may arise during the semester (this is also picked up under "Assessment" below).\\



\textbf{Employability and entrepreneurial skills}

In recent years the School of Mathematics has been looking at ways to embed employability and entrepreneurial skills into its undergraduate degree programmes. What Vince is planning with this new first year CfM module truly addresses this need. This is particularly exciting as it is aimed at our first year students and will hold them in good stead throughout their degree programme. Indeed the School should now look to build on this module to ensure further learning opportunities of this nature and increased employability skills during years 2 and 3.  Vince should be congratulated on putting together a schedule of lectures (weeks 1-6) in liaison with the entrepreneurial team from the Cardiff University Students’ Union.   The indicative order of these lectures does not however seem to reflect the likely needs of the groups over time, since for example the ‘Meeting Skills’ session would ideally come first whereas “Presentation Skills” that aren’t required until later on can be correspondingly scheduled later.   Vince should give more thought on this matter and arrange in an order that is more appropriate.

The seminars in weeks 7-11 by invited speakers are also an excellent idea and will reinforce employability and entrepreneurial skills as well as demonstrating research-led teaching (by exposing students to research activities within the School).  We discussed how external speakers might also be invited to increase awareness amongst the students of where programming and mathematics are successfully used in industry.

\textbf{Assessment}

The three-week reports, although summative, act in a more formative role to ensure satisfactory progress by each group and that they are approaching the project in a sensible manner. It also has the benefit of assessing the third learning outcome of CfM to \emph{have a basic knowledge of LaTeX} which is what the groups will use to submit their written work. The group presentation is an entirely appropriate assessment method for attaining the fourth learning outcome of CfM. I am entirely happy with the proposed marking criteria/framework that has been well designed and allows for all related ILOs to be assessed.

As discussed earlier, it will be important to foster/encourage good group dynamics. The current assessment scheme awards the same mark to all group members, regardless of actual relative contribution. This is always a tricky issue and one that Vince and myself have discussed on several previous occasions in relation to group assignments on MSc modules. Vince might wish to consider mechanisms to encourage/incentivise inclusivity such as reserving some of the available marks to individual performance or at least the possibility to differentiate marks within the group based on peer-review/reflection.


\section{Concluding remarks}

Below I sumarise ways in which several ILOs from PCUTL module 3 are being specifically addressed by Vince in the planning and delivery of MA1003:

\textbf {Integrate scholarship, research and professional activities with teaching and supporting learning [ILO 1]}

Inviting internal and external speakers will encourage the \textbf{integration of research and professional activities} to benefit the students' learning whilst \textbf{firmly placing employability and entrepreneurial skills within the module}. Adopting business terms such as `companies' and `project managers', which are closely akin for example to the popular TV show \emph {The Apprentice}, is an excellent way to incentivise students and foster such skills.

\textbf {Design and plan effective modules or clusters of sessions or programmes of study that facilitate quality learning and the achievement of appropriate learning outcomes by a range of learners [ILO 2]}

The nature of the planned group assignment facilitates independent research whilst the support mechanisms and additional sessions put in place encourage group working, research investigation, time-management and entrepreneurial skills.  In summary this is an \textbf{exemplary} programme of study that creates \textbf{multiple learning opportunities} for our students within their first year of the degree programme.  It permits the achievement of \textbf{learning outcomes by a range of learners}.  Indeed it would be nice to see a similar module in our final year.  Vince might well reflect on how skills acquired in this module can be built upon in other years, and discuss these opportunities with colleagues.

\textbf {Design and implement appropriate and effective assessment and feedback schemes using a range of methods that align with the tenets and principles of the Cardiff University Assessment Strategy and Feedback Policy [ILO 3]}

The proposed assessment and feedback schemes seem entirely sensible and appropriately aligned to the learning outcomes.  Clearly much thought has gone into planning the nature of the coursework and deliverables with a clear and transparent marking scheme.  Mixtures of summative and formative methods are to be used, with timely feedback.  I particularly like the concept of the ‘Grand Council’ in week 6 to facilitate collective learning as well.  Certainly in my opinion the proposed module is well aligned to the overarching principles of the Cardiff University Assessment Strategy and Feedback Policy assessment, namely that it is \textbf{valid, reliable and explicit}.

\textbf {Work with colleagues to enquire critically into an aspect of planning or assessment / feedback relevant to their context [ILO 6]}

The very fact that we sat down to discuss and critique this module, focussing on planning, assessment and feedback, contributes to this ILO! In addition I know that Vince contributed significantly to the module 3 group exercise, and consequently gained a great deal, and he has translated insights and findings from that exercise to the planning for CfM group work.\\

In summary, I am impressed with the thoroughness and conscientiousness planning that has gone into MA1003. The innovative methods employed by Vince are to be congratulated and I anticipate that Spring semester will be a great success and suitably challenging but rewarding for our students.

There are some aspects for further consideration and reflection as outlined above, namely: group membership/inclusivity, group contributions and equitable workload/marks, the ordering of the week 1-6 lectures, inviting external speakers from across different industries in weeks 7-11, and reflecting on ways in which this module can be built upon for years 2 and 3 and discuss within the School as required.

In closing I simply can't speak highly enough of what Vince is now achieving in his teaching within the School, resulting from continued reflected practice and exposure to methods and literature throughout PCUTL. It is truly awesome to witness and if I might have helped even in some small way during his journey so far, that is incredibly gratifying.  May he long continue to be an ambassador for innovation in teaching and learning, helping his colleagues be inspired too, continue to reflect on his own teaching and evolve practice as necessary, and ultimately to create further learning opportunities to benefit our students.


\end{document}
