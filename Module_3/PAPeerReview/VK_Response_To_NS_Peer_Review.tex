\documentclass[a4paper,12pt]{article}

\usepackage{hyperref}
\usepackage{graphicx}
\pagestyle{empty}

\title{PCUTL - Module 2:\\ Response to NS Peer Review}
\author{Vincent Knight}
\date{}

\begin{document}

\maketitle

In this document I'll respond to the peer review undertaken with Nikos Savva.


\section{Inclusivity}

\begin{quote}
``Vince placed a lot of emphasis in making sure that both his teaching methods and materials are largely inclusive (for further comments related to the available resources see comments below).

An important aspect in his teaching is the formation of student groups to work together both during contact-time and non-contact-time activities. In his lesson plan Vince commented on the formation of groups in such a way to ensure a mix of age, gender and cultural heritage. It is not clear to me how this task can be achieved and whether aspects of student characteristics such as academic performance and computer literacy are also taken into account which can also play some role in group dynamics."
\end{quote}

Nikos makes an excellent point as to the methodology used when choosing groups. This should have been better explained in my lesson plan. Our MSc course benefits from the presence of a Knowledge Transfer Office who builds a great relationship with the students throughout the year and for all group activities is tasked with creating inclusive groups. The KTO takes in to account all of the characteristics mentioned by Nikos and I'm confident that groups are inclusive.

\section{Resources and technology use}

\begin{quote}
``An abundance of resources will be made available to students mostly in the form of lecture notes and youtube videos.

More specifically, detailed sets of lecture notes are provided for both packages in a variety of formats. The notes include screenshots and code snippets, are engaging and are prepared in a conversational style, which can help students master the material more easily.

The work that went into preparing the online videos is truly remarkable. The videos, which are offered at different resolutions, are very informative and Vince explains clearly and slowly every step required to utilise various features of the software packages that would benefit students with no prior exposure to SAS and R. Most importantly, making such videos available gives students the opportunity to go over the material at a pace that is most comfortable to them.

I also liked the idea to make the lecture notes available on Google Docs, so that students can make amendments. However, will the content be  moderated by Vince to ensure the accuracy of the amended information, or will the it be entirely left at the students' discretion? Given also the tendency of students to learn what the instructor gives them as lecture material, I could foresee some reluctance in actually stepping in and making their own contributions. Another possibility worth considering is perhaps posting the best solutions from the group assignments instead of posting his own solutions, which might also be complemented with slides from their presentations (if available). This might give additional incentives to students to invest more time in their group activities in order to polish their work further."\end{quote}

Nikos's kind words regarding my resources are greatly appreciated. I have indeed spent a lot of time on them! I had not given much thought to moderating the use of Google Docs. This is mainly due to the fact that having used Google Docs previously students did not engage at all and did not use it. As such I'm still making it available to students and on some level would be `glad' if they wrote anything at all: even if it is incorrect! Having said that, after chatting to Nikos about this I'll be sure to keep an eye on the Google Document. In fact if a student does write something that isn't completely accurate, it would give me a good opportunity to address the issue in class.

\section{Non-contact-time activities}

\begin{quote}
``Students are provided with ample opportunities to engage and practice outside the classroom, which are likely to appeal to a variety of learning styles. Student collaboration features rather strongly in non-contact time teaching, by having students work on challenges and by setting up an online community where students can participate in discussions.

A minor comment I would like to make is that since contact sessions are designed around non-contact time activities, it might be good if students are informed how much time they are roughly expected to invest in preparation for each session. This can be helpful in a number of ways, especially in scheduling the meetings so that students allocate sufficient time to work as a group.

Lastly, during the contact sessions, students are expected to report their findings to the rest of the class. Again, some information on what is expected of them (e.g., the duration of the presentation, whether software use is required etc.) might help relieve some of the anxiety that is natural for students who do not have much experience with public speaking, or for students with weaker communication skills."
\end{quote}

I again agree with Nikos and in fact plan to clearly explain to the students the amount of work required as well as what I expect from them with regards to presentations. This I will in fact leave up to the students, if they choose to simply demonstrate their code and/or use a complicated piece of software I do not mind.

\section{Contact-time activities}

\begin{quote}
``Teaching how to use a software package essentially involves transmission of a lot of procedural knowledge, i.e. students need to learn the sequence of steps or commands required to perform certain functions. Vince, however, tries to ensure that this is not a one-way process, but a process that involves continuous feedback, interaction and practice, which I find rather commendable. It is apparent that the lesson plans are primarily student-centred and are based on teamwork. This type of classroom design shifts the focus from the teacher to the various student groups and the teacher acts more like a moderator than a lecturer. To achieve this, students need to be more actively involved in their learning, while at the same time they develop their presentation and teamwork skills.

The class consists of MSc students, so it is natural to assume that the audience is sufficiently motivated. However, Vince also gives additional incentives to engage in classroom activities, by making classroom participation count towards the final grade. I also agree with Vince that a holding a debriefing session prior to the first session is essential to inform students what is expected of them, explaining also the different approach he intends to adopt in organising his sessions. Sometimes, however, despite our best intentions, students might be reluctant to participate and engage fully in discussions. I am thus wondering whether Vince had thought of any proactive measures to ensure the smooth running of the contact sessions.

A final comment I would like to make is that a four-hour session can be rather tiring for both the students and the lecturer. While the diversity of activities will perhaps help students stay alert for a longer period of time, it is also natural to expect that student engagement levels/attention span etc. will tend to decline with time. Hence, since the discussions on R are always planned to take place during the second half of each lesson, the material on R might appear to be more difficult. Hence I would suggest that Vince can perhaps consider varying the order at which the sessions are delivered so the lesson is not always about SAS early in the morning and about R just before lunch. Moreover, the design of the various tasks to be performed during each session could also take into account these factors."
\end{quote}

Nikos hits on an excellent aspect of this module design: the need for student engagement and buy in. As described I have designed various incentives to try and take this in to account. Also, the debriefing session will also hopefully be helpful but nevertheless if students do not engage during contact time I will need to adapt.\\

Ultimately every piece of contact time must be used to reach the described ILOs. I will hopefully act as a passive observer as the students guide themselves towards these ILOs but might well have to intervene. Either because students miss a particular ILO or indeed because engagement is so low that no ILO will be reached. As such I will be prepared to step in and potentially deliver a classic lecture on some or all of the topics.\\

Nikos's other comment regarding the length of the day is an important one. I have purposefully included breaks and intend that students might be able to recover slightly during the lab sessions. Nikos makes an excellent suggestion which is to swap SAS and R throughout the course. I had not thought of that and appreciate it being suggested. It will be very helpful.

\section{Final thoughts}

\begin{quote}
`` It is quite apparent that Vince put a lot of effort in this bold attempt to teach inclusively in a flipped classroom approach. His hard work and dedication are strong indicators that he will succeed!"
\end{quote}

These kind words are appreciated. I very much gained from this peer review as Nikos was able to confirm some of my ideas as well as point out potential weaknesses.



\end{document}
