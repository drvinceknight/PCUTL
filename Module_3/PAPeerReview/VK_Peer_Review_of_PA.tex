\documentclass{article}

\usepackage{fullpage}
\usepackage[parfill]{parskip}

\title{PCUTL - Module 3:\\ Peer Review of Phil Anderson}
\author{Vincent Knight}
\date{}

\begin{document}

\maketitle

\section{General comments}

The subject of this peer review was module design. In particular the re-design undertaken by Phil for `EN2705: POWER ENGINEERING 3' a second year module.

Our discussion began with the fact that Phil has his hands bound somewhat and is not only unable to make changes he wants due to university regulations but also due to accreditation. This is an aspect of course design that I have not had to worry about as there is no accreditation body for mathematics.

Due to these restrictions the changes Phil has planned are to be done in stages.

I will discuss my understanding and opinions on these modifications and conclude with some queries and concerns. Importantly, before beginning I would like to state how impressed I am with Phil's engagement with the underlying pedagogic notions linked to learning outcomes of his module.

\section{Linking syllabus to learning outcomes}

It seems that EN2705 has evolved over time to a module that (in my non-expert) opinion no longer suits the purpose for which it is designed. Phil presented me with the previous (prior to Phil's modifications) module description and it is quite interesting to see that the learning outcomes do not seem to be matched to the syllabus.

As a result Phil has redesigned the module with an alignment between learning outcomes and syllabus. The result of this seems (to my eyes) to be a well aligned module. Furthermore, the analysis of the learning outcomes in view of Bloom's taxonomy was a great thing to see. It now seems that the course is much more level appropriate.

If I was to voice one minor concern it would be how this now fits in the rest of the programme? Perhaps the slight change of syllabus will have effects on the rest of student progression?

\section{Pedagogic theory}

Another aspect of Phil's modification includes a great consideration of the variety of learning styles he is likely to come across. In particular I approve of his dissemination of a full set of notes at the beginning of the module. This allows students with difficulties to read ahead in time for the lecture but also students who want to speed ahead to do so.

Further modifications which are in line with my preferences for pedagogic practice include a reduction in the number of lectures and an increase in the number of tutorials.

This has all been very well thought out and Phil should be commended on this modifications being based on sound pedagogic theory and not arbitrary decisions.

\section{Further plans}

All of the above are the first phase of Phil's planned changes. The second phase is noteworthy and I'd like to spend some time discussing it here. In particular the proposed changes will need to gain approval from the accreditation body.

Phil plans to involve more industrial talks (in his modified course, an industrial talk is given). I was surprised to hear that when this talk was given it increased student engagement. In my experience I have often seen students disengage with anything that is `not on the exam'. Perhaps this was due to the speaker himself and as such I'd suggest that Phil ensures that if more speakers were to be used that they should be `good speakers'. I plan on using some industrial speakers in the second half of a module I am teaching and might seek some advice from Phil on the subject.

One final aspect in Phil's future plans revolves around the change of locus of information delivery. Phil suggested the use of videos and/or other means. I commend Phil and fully back this idea. I would just raise a capacity issue related to this plan. Designing videos for this purpose takes a very large amount of time.

\section{Suggestions and conclusions}

I thoroughly enjoyed this discussion with Phil who seems to have a great number of ideas. Importantly, it seems that a lot of what Phil has decided to do is based on official subject standards: QAA Benchmarks for Engineering. Phil has obviously given thought to `what it means to be an engineer' and this even led us to a discussion on what would be an ideal assessment. In particular Phil talked about the possibility (`pipe dream') of using portfolios instead of exams. I don't know why but I feel cautious about this. I completely understand that this discussion was hypothetical and so I will continue to talk about this in that vain. I feel perhaps that Phil would face a lot of barriers to implementing a full portfolio style of assessment. Having said that perhaps I am just clinging to `the way things have always been done' and it is important to constantly question methods of teaching and learning.

There are other barriers of concern that come with certain changes that Phil plans on implementing. In particular I wonder how Phil will balance student expectations and/or the pressures related to the NSS standards expected by the University. This is a difficult balance for us all as educators. Students might indeed `prefer' certain pedagogic models whilst they are not necessarily the best possible model for them. Evidencing the benefits to the students of any innovative pedagogic model is a difficult task.

Overall I think that Phil is doing a great job with this redesign.

\end{document}
