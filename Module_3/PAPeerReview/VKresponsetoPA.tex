\documentclass{article}

\usepackage{fullpage}
\usepackage[parfill]{parskip}

\title{PCUTL - Module 3:\\ Response to Peer Review by Phil Anderson}
\author{Vincent Knight}
\date{}

\begin{document}

\maketitle

\section{Summary}

\begin{quote}
``Vince has designed (and is currently implementing) a new module, `Computing for Mathematics', giving a great deal of attention to the formative elements and feedback.  The focus of this PRLT is the feedback mechanisms employed.  The module structure is highly innovative with the lectures being very much reactive to the difficulties faced by students in the lab session.  This, however, is only one form of feedback in the course.  The lab sessions are designed to allow rapid feedback through his `tickables' approach which, interestingly, acknowledge an informed attempt at a particular element and not necessarily a correct answer.  Vince also uses videos to provide direct feedback on the exercises.  Finally he has begun to introduce a fixed time when he will be available for students to visit his office for one to one discussion.

The approach taken to feedback here is the most thorough I have seen and meets all of the requirements of Cardiff University's Feedback Policy, particularly in that the feedback is extremely timely, appropriate to the learning, continuous and suited to a wide range of students' needs.  The sheer range of feedback opportunities should guarantee that all students will find an approach which meets their needs.

It is a reflection of this thoroughness that this module will inevitably take up a lot of Vince's time since so much of the work is bespoke although this workload will reduce with each new cohort as I am sure many of the issues will resurface every year.  The size of the cohort (165 students) also presents scheduling difficulties for the lab sessions (as there are approximately 16 students in each lab session and all must be completed before the lecture).''
\end{quote}

I really enjoyed talking about my plans for this module with Phil as he was able to query quite a few of my plans. This led to a great discussion about feedback and pedagogy. Phil's comments about my thoroughness are very much appreciated as it has been a long process of constant reflection and re-evaluation to obtain the module design as it is.

Phil raises a valid concern here about how much time this will take. As the module has been running already I have found it slightly challenging to prepare my reactive lecture although as the course runs in future years I am sure I will be able to better preempt the difficulties of students.

During our discussion Phil raised certain comments and suggestions that allowed me to better

\section{Comments and Suggestions}

\begin{quote}
``Vince has clearly considered and planned this module, and in particular the feedback mechanisms, extremely well and I am sure it will be a success.  He appears to be committed to the use of flipped classrooms and the benefits that brings in terms of feedback opportunities and deeper learning.  It is refreshing and inspiring to see such commitment to innovative teaching and I applaud both Vince and his school's willingness to take risks (albeit risks backed up by sound pedagogic research) with teaching methodologies.

I suggest that consideration should be given to a few minor points.

Firstly care should be taken to ensure that students do not get bogged down in achieving perfection in the tickables and that they cover the breadth of the work required.  This will require the lab supervisors to have a good understanding of the rationale and the ability to communicate this to the students.

The module is designed to cater for students who have no previous programming experience, those who come in with significant experience should still find sufficient work to challenge them.  It will be difficult to manage the assessments such that this happens whilst still ensuring that top marks are achievable for all students.

The feedback mechanisms appear quite time intensive for those involved, consideration should be given to introducing efficiencies to this process.  Will this naturally occur with future running of this module?  Can larger groups be managed in lab sessions?  Perhaps if students were not all tackling the same exercises simultaneously an element of peer teaching could be introduced.

I hope that Vince will have the opportunity to further report on his findings either through module 4 or through a paper reviewing the findings and in particular the student perceptions of the feedback received.  ''
\end{quote}

I appreciate Phil's kind comments and in particular his understanding of my willingness to take risks. Given the amount of research undertaken and the multiple locations of evidence for the effectiveness of my methodologies I do not consider this as risky endeavour (although it would have indeed been easier to use a classical approach). Instead, I feel that I have simply asked myself: `if I had no preconceptions with regards to teaching and learning how would I deliver this course?'.

Phil's particular suggestions and concerns are all gratefully received:

\begin{itemize}
    \item Ensuring a sound understanding of what is expected of the students is indeed important. I have to ensure that I communicate not only to the students but also to the tutors. This is something that I have had to deal with during the first few weeks of the course: some tutors had not quite understood what was expected of the students. After a couple of meetings and discussions this has been addressed and I am aware of how to ensure this is all well communicated to the tutors and students next year.
    \item Catering for students who are very comfortable in the course is something that I have also thought of. Although I am glad that Phil mentions it here as I've been concentrating on evidencing that students with difficulties won't be left behind by my teaching approaches. There are various programming projects that I have put on offer to students through the recommended reading list that contains more advanced books.
    \item Phil also raises very valid concerns as to the workload issues associated to the tutors. Having larger labs is not a possibility at the moment (due to classroom size in the School of Mathematics) and it is already quite intense for the tutors. Based on my discussion with Phil I will think about the possibility of having extra tutors to ensure that a more efficient process. Peer instruction is already encouraged throughout these sessions: for example if a particular exercise is causing difficulties to students, they are encouraged to work in groups to solve it.
    \item It is very encouraging that Phil would like to hear of how my delivery works out. I plan on evaluating its effectiveness over a series of years and so will hopefully be able to report on my findings, either informally or through a more formal approach such as a publication.
\end{itemize}

\bibliographystyle{plain}
\bibliography{../Essay/educationresearch.bib}
\end{document}
