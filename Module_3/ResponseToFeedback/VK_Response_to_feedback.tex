\documentclass[a4paper,12pt]{article}

\usepackage{fullpage}
\usepackage[parfill]{parskip}

\title{PCUTL - Response to Module 2 Feedback}
\author{Vincent Knight}
\date{}

\begin{document}

\maketitle

In this document I will respond to each element of feedback from my module 2 portfolio.

\section{Learning Outcome One}

\begin{quote}
    ``I thought a strength of this submission Vince was your engagement with pedagogic literature. Your reading (and Maths literature in particular) helped you locate your approach but did so in an enquiring manner. Very good work!''
\end{quote}

Thank you for your kind words. I felt that I did invest a lot of time and effort going engaging with the literature and this is something that I have perhaps done more of with in my third portfolio. Furthering my understanding of the research related to my pedagogical approaches.

I enjoy reading this literature and plan on doing so in the future.

\section{Learning Outcome Two}

\begin{quote}
    ``Well! Your plans are radical! Carefully thought out plans here Vince. Do be careful to ensure that no student is left behind in your wish to radicalise your teaching and learning.''
\end{quote}

This is a very valuable piece of feedback. I think that I must be very conscious of not `being radical for the sake of being radical'. I feel that I have spent more time during this portfolio/module design carefully putting in to place traditional alternatives so that no student is left behind.

\section{Learning Outcome Three}

\begin{quote}
    ``You have used data from a number of sources which was good to note. Your new Module will, of course need careful evaluation. You might find it helpful to have a disinterested person involved here... Your mentor will, of course be able to help.''
\end{quote}

The evaluation of the module (through the usual feedback forms) has shown that students reacted favorably to it. Furthermore, the marks obtained were also of a high standard and students performed well against the intended learning outcomes. This coming year, due to my workload a post-doc will be teaching the module which will allow me more time to evaluate it.

\section{Learning Outcome Four}

\begin{quote}
    ``This was very well done Vince. I could see the links that you were making to your reflection and also to literature. Very useful set of ideas to take forward. M3 will offer you some space to attend to some of these.''
\end{quote}

I have been able to expand on some of the ideas set forth but sadly not on all of them. I have again set myself an ambitious plan for module 4 which I look forward to doing.

\section{Programme Values}

\begin{quote}
    ``Very comprehensively underpinned.''
\end{quote}

Thank you.

\section{Engagement with the UKPSF}

\begin{quote}
    ``Good engagement with the UKPSF also.''
\end{quote}

Thank you.

\section{General comments / thoughts for module 3}

\begin{quote}
    ``Thank you for completing your self assessment form Vince. I think that you have indicated your willingness to reflect/evaluate and act on this. Very well done. What I would suggest is that you now allow your plans to bed down. Ensuring these innovative plans work will take time and effort especially the aspects highlighted by Nikos i.e. preparing students for the IBL/flipped classroom approach! This will be a new way of learning for many... your reading on pedagogic theories will help with this.

    Very good work! Well done.''
\end{quote}

Thank you again for your kind words. I agree with you completely with regards to needing to let my `plans bed down'. I am indeed planning on doing this for MAT013 (the module discussed for the portfolio relevant to this feedback) and will be able to step aside and watch the post doc lead the module so as to better assess.

\end{document}
