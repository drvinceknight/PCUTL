\documentclass{article}

\usepackage{fullpage}
\usepackage[parfill]{parskip}

\title{PCUTL - Module 3:\\ Response to Peer Review by Peter Burnap}
\author{Vincent Knight}
\date{}

\begin{document}

\maketitle

\begin{quote}
``Vince chose to focus on feedback in the peer review, with a particular emphasis on the flipped classroom pedagogic model that he is using to deliver a new first year module. From a feedback perspective, the flipped model ensures a continuous feedback loop between the teacher and their learners. It moves from content being delivered in contact sessions with practice at home, to the content being studied outside the classroom with structured practice sessions during contact sessions. Context and theory is discussed (as opposed to delivered) in interactive lecture slots.''
\end{quote}

This was a valuable peer review as Pete has a lot of experience with teaching programming. Given the nature of my module it is more akin to a computer science course than a mathematics course and as such his opinion was appreciated.

\begin{quote}
``My first question was whether the students (particularly in year 1) would be willing and able to undertake the understanding of content outside the classroom so that a useful and interactive discussion could take place during the timetabled  contact session. The is a clear risk here that some students will not engage with this approach and will struggle. While some students who do not attend lectures still achieve a pass overall by learning from the course notes, one would expect that in a flipped classroom students may not have access to the same type of materials as a traditional lecture. The answer to this question was very interesting. Vince has devised a series of `tick box' evaluation measures that are very quick to measure. Students are expected to demonstrate their achievement of relatively simple skill-focussed tasks in lab sessions (less than one minute per student) to a lab tutor who “ticks the box” for this student and records each students’ achievement on a Google Doc spreadsheet. This is an excellent use of collaborative technology as it allows Vince to get an “at-a-glance” view of student progress. He showed me the spreadsheet and it is clear to see where the minority of students have failed to engaged. The motivation to engage is that 10\% of the overall mark would be deducted for non-engagement with lab exercises.  Vince was clear that only extreme non-engagement would result in the loss of 10\%.  On reflection it may be worth him considering what would happen if one day questions arose as to how the engagement would be quantified – i.e. how much do I `have' to do before I achieve the marks?''
\end{quote}

I enjoyed talking over the tickable system with Pete and the incentivisation that he mentions is at the forefront of my concern with this. The point he raises about how to quantify `engagement' is an extremely important one. I had not considered this until the point was raised by Pete. Currently tutors are told to use their subjective judgement but perhaps in future years I will need to put in place guidelines that would not only help tutors but also be transparent to the students. Furthermore in future years I will be using 2nd year students as tutors and so I think this approach could be helpful.

\begin{quote}
``A further question was related to how students who clearly didn't engage were managed. From the spreadsheet Vince is able to identify students who have not engaged during any week/month and send them a targeted email asking them to come and see him to discuss their non-attendance/engagement with the lab tasks. They would then be expect to come and see him during scheduled `office hours' where he has set aside two hours per week for students (engaged or not) to drop in and discuss any issues they are having with the work. So far this has been successful but it is worth considering what would happen if 30-40 students all turned up to his office for this. The office hours are supplemented with an excellent provision of `feedforward' videos where anticipated problems relating to each `tickbox' task are discussed and guidance on how to demonstrate achievement of the anticipated learning outcomes of the task. This provides a range of options to students who may or may not want to speak to the lecturer on an individual basis, or who would prefer to work at night etc. It is another example of innovative use of technology in support of inclusive and diverse learning methods.

Thus Vince has provided a range of opportunities for feedback that offer different options to students based on their desired method of interaction. Feedback is given in contact sessions (lectures and labs), and in one-to-one sessions during office hours. Feedforward is also given via video snippets.''
\end{quote}

Thankfully the numbers for these students are quite low and I seem to be able to manage with both of them through office hours, however I need to entertain the possibility of a large number of students needing assistance. If this occurs I imagine that the students would have a common problem and as such I could possibly take them to a lab and give an impromptu lecture. This is something I need to think about further. Pete's kind words with regards to the videos are also appreciated.

\begin{quote}
``One further point on feedback was that of inclusivity of people with impairments. Vince noted that to demonstrate achievement of `tickbox' tasks, the students need to explain their work to the lab instructor. He found that in one case the student felt uncomfortable and unable to do so, and he quickly identified an alternative method whereby demonstration was achieved by communicating with another lab tutor (not the main lecturer). I have found that the involvement of another tutor who is seen more in a supporting role as opposed to the main lecturer is an important inclusion in an international classroom where, for some cultures, it is not normal for the student to question the teacher. This stems intellectual debate somewhat in a discursive environment so involving an informed tutor can be useful to include these students.  ''
\end{quote}

Pete raises a very important point that I have not been able to address fully in my portfolio so far (I have discussed some issues with regards to inclusivity on socio-economic grounds in the concluding portion of my essay). The one anecdote that Pete mentions I feel was handled well and I have since discussed this with the student in questions and have agreed to carry on with the current arrangement. I have made clear that if the student ever does feel like talking directly to me they are very welcome to. With this question Pete has ensured that I also consider further aspects with regards to disability. In my reactive lectures I try to ensure students stand up and take part in various role playing exercises: I need to be cautious with this if and when students with disabilities affecting their mobility are in my class. Of further concern will be the provision of computing facilities in the case of students that are perhaps blind. I should also ensure I make efforts to include sub titles for my videos in the case of students with hearing difficulties. These are all reasonable adjustments that I will make if and when the need occurs.
\end{document}
