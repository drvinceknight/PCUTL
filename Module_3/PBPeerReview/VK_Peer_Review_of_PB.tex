\documentclass{article}

\usepackage{fullpage}
\usepackage[parfill]{parskip}

\title{PCUTL - Module 3:\\ Peer Review of Peter Burnap}
\author{Vincent Knight}
\date{}

\begin{document}

\maketitle

\section{General comments}

The subject of this peer review was `real-time' formative feedback for a 20 credit level 7 module called: Security Techniques. In particular we discussed the `ethical hacking' aspect of the module taught by Pete.

Pete discussed not only aspects relevant to this portfolio but also some aspects of the module content which I found really interesting!

The MSc program Pete teaches is quite similar to the MSc course on offer here in so much as that it is delivered through intense single day bursts. This in itself brings with it various challenges that I feel that Pete has been able to address very well. I will discuss this in general before giving some suggestions and conclusions.

\section{Handling a variety of abilities}

Pete is faced with students from a variety of disciplinary backgrounds. As such he has the difficult task of teaching some quite complex computational techniques to some students who have no (or very little) knowledge of communicating with a computer. This draws some interesting parallels with the MSc in OR offered at the School of Mathematics where some of our students do not have basic knowledge of mathematics. We have addressed this through the use delivery of basic knowledge during the induction week of the course. Pete however has addressed this by a careful and timely iteration through basic concepts up until all students are up to speed.

\section{Assessment of individual learning}

There is a constant and timely feedback loop in real time between Pete and his students, this is mainly achieved through small group discussions after tasks have been completed. This has been very well thought out and furthermore allows for group and peer learning. The feedback is also able to be given on an individual level as Pete ask individual questions to all members of the groups. I have a slight concern as to ensuring that students recognise that this is indeed feedback but I will return to this at the end of this peer review.

\section{Disengaged groups}

One aspect with regard to the group is the potential for groups who have had their discussion with Pete at the beginning of the session. Chatting with Pete, this is something that has already been considered and he will be having some further pieces of work for students to not be bored. Perhaps a further possibility would be to encourage students who have finished their work to engage in peer instruction (helping students who have not finished yet).

\section{Suggestions and conclusions}

One aspect of Pete's planning that I have not discussed yet is how he aligns student expectations with their learning experiences. A session at the beginning of the class involves gathering student expectations from the students in class and comparing these to the ILOs. I think this is a great way of doing things and I will consider implementing something similar in my own teaching.

One final aspect of consideration and relevant to Cardiff University's goals achieving high NSS scores is the importance of ensuring that all the feedback that is taking place throughout Pete's instruction is recognized as such. I think Pete is doing a great job ensuring there is a timely and relevant feedback loop in place. Often though, feedback is interpreted by students as implying some written feedback. Ensuring that students recognise feedback for what it is, is something I suggest Pete considers carefully.

\end{document}
